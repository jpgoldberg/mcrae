\documentclass[12pt]{article}

\usepackage{amsmath}
\usepackage{booktabs}
\usepackage{csquotes}
\usepackage[standard]{ntheorem}

\providecommand{\land}{\wedge}
\providecommand{\lor}{\vee}
\providecommand{\lif}{\rightarrow}
\providecommand{\liff}{\iff}
\providecommand{\lnot}{\neg}
\providecommand{\lfalse}{\bot}
\providecommand{\lall}{\forall}
\providecommand{\lis}{\exists}
\newcommand{\rel}[1]{\ensuremath{\mathop{\mathsf{#1}}}}

\newcommand{\rlan}{\rel{Ancestor}}
\newcommand{\rlpt}{\rel{Parent}}
\newcommand{\rlapple}{\rel{Apple}}
\newcommand{\rlfruit}{\rel{Fruit}}
\newcommand{\rlbelieve}{\rel{Believe}}

\newcommand{\rlatheist}{\rel{Atheist}}
\newcommand{\rlwa}{\rel{WeakAtheist}}
\newcommand{\rlagno}{\rel{Agnostic}}

\newcommand{\rllove}{\rel{Love}}
\newcommand{\rlhate}{\rel{Hate}}




\begin{document}

Steve McRea has been attempting to present what should be a simple and uncontroversial logical argument, but he has run into substantial resistance.
I believe that there are two reasons for the resistance, but before discussing those, I want to first make sure that I (and perhaps others) properly understand his argument as he intends it.

\section{Logical notation}

The argument that he has given is presented using propositional symbolic logic.
In recasting it, I will sometimes resort to first order predicate logic, mostly to make it easier to talk about certain parts of the argument, even if it isn't strictly necessary. The notation I will be using for propositional logic is shown in table~\ref{tab:prop}. 

\begin{table}
    \begin{center}
    \begin{tabular}{cc}
        \toprule
        Symbol & Word \\
        \midrule
        $\land$ & and \\
        $\lor$  &  or \\
        $\lnot$ & not \\
        $\lif$  & implies \\
        $\liff$ & if and only if \\
        $p, q, r, \dots$ & various propositions \\
        \bottomrule  
    \end{tabular}
    \caption[Propositional logic symbols]{Symbols for propositional logic with hints at what they kinda-sorta mean in English}\label{tab:prop}
    \end{center}
\end{table}

I am not going to teach symbolic logic, but I will give a couple of examples to help people read those. If we let $p$ stand for the proposition “Jones owns a Ford” and we let the $q$ stand for the proposition that “Brown is in Barcelona”, we could represent “Jones owns a Ford or Brown in in Barcelona” with $p \lor q$.
We can represent more complicated things by building up expressions from the symbols.

Propositional logic only allows for talking about the relationships between simple proposition without being able to look inside them.
But suppose we want to talk about the relationship between notions like $\rlapple$ and $\rlfruit$. So say that some thing, $t$ is an apple we would say $\rlapple(t)$.

We might want to be able to say that if something is an apple it is a fruit.
For this we need a way to talk about all things. And the notation for that is the universal quantifier ($\lall$), which you can read as “for all.” So one way to represent “all apples are fruits” is to say that if for anything that it is an apple that this is also a fruit. And that is what we have in expression \ref{exp:allapple}.

\begin{equation}\label{exp:allapple}
    \lall x \rlapple(x) \lif \rlfruit(x)
\end{equation}

$\rlan$ and $\rlpt$. So if we want to say that Alice, $a$, is a direct ancestor of Bob, $b$, we could notate it with something like “$\rlan(a, b)$”.

We might also want to say things like an apple exists, then a fruit exists.
We can do that using what is called the existential quantifier ($\lis$), which you can read is “there is.” This is exemplified in \ref{exp:isapple}.

\begin{equation}\label{exp:isapple}
    \lis x \rlapple(x) \lif \lis y \rlfruit(y)
\end{equation}

Note that \ref{exp:isapple} doesn't say that the thing that is the apple is the fruit.\footnote{Don't worry about this. It just means I should have found a better example that doesn't add in that extra confusion.}

\section{Steve's first argument}\label{sec:first}

Steve takes issues with Atheists who like to state that Atheism is a lack of belief.\footnote{I also think that the claim that “Atheism is a lack of belief” is silly, but for very different reasons than Steve does. I am, however, trying to present Steve's argument in this section.}
The guts of the argument is that "lack of a belief about the existence of a god or gods is logically equivalent to agnosticism.

Let's follow Steve's notation and use $g$ to denote the proposition that god (or gods) exist.
I am going to depart from his notation by treating \rlbelieve\ as a two-place relation. So $\rlbelieve(x, p)$ is a way to say that individual $x$ believes proposition, $p$. This departure doesn't change anything of substance (yet), but is perhaps clearer as it corresponds more closely to English “believe”.

What I most notably do differently is make explicit and clear how notions like (strong) Atheist, weak Atheist, and Agnostic are represented. Again, I am attempting to follow Steve's intent.

In Definition~\ref{def:satheist} we define  \rlatheist\ in terms of \rlbelieve\ and $g$. Note that this definition is intended to correspond to what some call a strong Atheist.

\begin{definition}[Atheist]\label{def:satheist}
    An individual, $x$, is an atheist if and only if they believe that the proposition that $g$ that god exist is false.

    \[
        \lall x \rlatheist(x) \liff \rlbelieve(x, \lnot g)
    \]
\end{definition}

The weak Atheist, by comparison neither believe that god(s) exist nor believe that god(s) don't exist. And that is what we intend to capture in Definition~\ref{def:weak}.

\begin{definition}[Weak Atheist]\label{def:weak}
    An individual, $x$, is a weak atheist if and only if they neither believe the proposition, $g$, that god(s) exist nor the proposition, $\lnot g$, that god(s) don't exist.
    \[
        \lall x \rlwa(x) \liff
          \lnot\left[\rlbelieve(x, g) \lor \rlbelieve(x, \lnot g)\right]
    \]
\end{definition}

Although we won't be using this, I will toss in the definition (\ref{def:theist}) for a theist, to server as a simple example of (and provide practice) for reading these sorts of definitions.

\begin{definition}[Theist]\label{def:theist}
    An individual, $x$, is theist if and only if they believe the proposition, $g$, that god(s) exist.
    \[
        \lall x \rel{Theist}(x) \liff \rlbelieve(x, g)
    \]
\end{definition}

Finally, we come to the definition of agnostic within this system. They don't believe that god(s) exist and they don't believe that god(s) don't exist.
This we attempt to capture in Definition~\ref{def:agnostic}.

\begin{definition}[Agnostic]\label{def:agnostic}
    An individual, $x$, is a agnostic if and only if they don't believe proposition, $g$, that god(s) exist and they don't believe the proposition, $\lnot g$, that god(s) don't exist.
    \[
        \lall x \rlagno(x) \liff
          \lnot \rlbelieve(x, g) \land \lnot\rlbelieve(x, \lnot g)
    \]
\end{definition}

Steve's argument (when recast this way) is that \rlwa\ and \rlagno\ are logically equivalent.
And in that he is perfectly correct. We can prove that they are logically equivalent using only the tools of propositional logic.
At a high level there are two steps to the proof. First we use (one half of) De~Morgan's Rule to show that $\lnot \rlbelieve(x, g) \land \lnot\rlbelieve(x, \lnot g)$ is the logically equivalent to $\lnot\left[\rlbelieve(x, g) \lor \rlbelieve(x, \lnot g)\right]$.

\begin{theorem}[De~Morgan's Rule 1]
    For any propositions $p$ and $q$ 
    \[
        \lnot\left(p \lor q\right) \liff \lnot p \land \lnot q
    \]  
\end{theorem}


The second step is to show that if \rlwa\ and \rlagno\ are equivalent to the same thing then they are equivalent to each other. That is, if $p$ is equivalent to $r$ and $q$ is equivalent to $r$ than $p$ is equivalent to $r$.\footnote{I could provide the full proofs of De Morgan's Rule and of the transitivity of equivalence, but doing so does not help illuminate the argument. Also, I would have to introduce the proof notation.}

From the above definitions and simple facts about propositional logic we can conclude \rlwa\ and \rlagno\ are the very same thing. The argument is sound and valid in logical terms.

\section{The objections}\label{sec:object}

Although the argument in section~\ref{sec:first} is logically sound, it fails to convince many\footnote{Well, perhaps not “many,” but many of the few who have been involved in the discussion.}
that agnosticism and weak atheism are equivalent.
It proves that weak atheism \emph{as defined} in definition~\ref{def:weak}
and agnosticism \emph{as defined} in definition~\ref{def:agnostic} are equivalent.
But if definition~\ref{def:agnostic} fails to capture some important and relevant aspects of what we mean by “agnosticism” then the proof of equivalence doesn't tell us anything interesting about what we are interested in.

I believe that definition~\ref{def:agnostic} fails to capture what people mean when they say that they (or someone) is agnostic.
Unfortunately, the logical frameworks that we've been using do not really have the capacity to capture the relevant aspects of the meaning of agnostic.
In fact, it may be the much of the dispute and confusion over Steve's argument has been an indirect consequence to trying to model agnosticism in a logical system that unequipped to represent it.
Furthermore, I feel that Steve's original presentation failed to make the definitions of weak atheist and agnosticism sufficiently explicit.
As a consequence, a number of people knew that they didn't like his argument, but were hard-pressed to identify their objection.
My goal in this section is to identify that objection. 

It is reasonable to say that people who explicitly consider themselves agnostic have considered the question of the existence of god(s), and so for this discussion it is fair to limit agnosticism to such people.
Later we may return to people who have not considered the existence of god(s), but the arguments about these terms are all among people who explicitly claim to be agnostic, (strong) atheist, or weak atheist.

\subsection{States of mind}

We are talking about beliefs, and beliefs are states of mind.
Let's use the following, contrived, story to set the scene for a discussion of belief states.

\begin{quote}
Suppose Alice lives in a city with two competing taxi companies.
The Green Cab Company and the Blue Cab Company. 
heir cabs are colored as expected. Suppose one misty night she sees someone who just robbed a store depart in a cab.
It appeared blue to her, but it was night and misty.
Bob, a police investigator, wants to know the probability that it really was a Blue Cab.
He weighs the evidence. He tries to get a sense of how often people mistake Blue Cabs for Green Cabs or Green Cabs for Blue Cabs under those visual conditions. He looks at the relative numbers of Blue Cabs to Green Cabs in the city.
After weighing the evidence he concludes that the chances it is a Blue Cab is 50\% and the chance that it is a Green Cab is 50\%.
Carol is another resident of the city, but she is entirely unaware of the the incident or any question about the color of a particular cab.
\end{quote}

In the above story Bob has considered what kind of cab it was and remains fully uncertain as to which of the two possibilities it was.
Bob, even after careful consideration, is Cab Color Agnostic.
Carol on the other hand lacks any belief about the color of the cab because she's never been asked or put in any position to consider the case.
Our intuitions are that Carol's belief state regarding the color of the cabs is markedly different than Bob's. 

We can very reasonably say that Bob is doubtful about the color of the cab from the incident. It would, however, be peculiar to refer to Carol as doubtful. Roughly speaking, Bob has beliefs about something and is doubtful about it. Carol lacks beliefs about it, and has no relevant beliefs to be doubtful of.
Bob and Carol have very different states of mind.

There are systems of logic, specifically tailored to dealing with belief and degrees of certainty, that can represent the distinction between Bob's and Carol's state of mind.
These allow people who study natural language semantics to treat belief and doubt formally, but it would take way too much time to introduce one of those theories.
Fortunately, we don't need to worry about how we represent the difference between Bob's and Carol's belief states;
all we need to do is acknowledge that they are different.

Whatever the differences are between Carol's belief state and Bob's belief state, those differences are parallel to the differences in belief states between someone who lacks any beliefs about the existence of god(s) and someone who has considered the question but remains uncertain.
When someone says they are an agnostic, they are saying that they have considered the existence of god(s) and remain uncertain.
The agnostic does believe things, they just believe them with a substantial degree of uncertainly. The agnostic can say, “I am uncertain as to whether god(s) exist.”
Carol from our example cannot say, prior to learning of the incident, “I am uncertain about the color of the cab the robber fled in during that incident that I don't even know occurred.”

\subsection{There's glory for you}

This distinction between being uncertain about something and not having a belief about it
is not represented in the difference between definitions \ref{def:weak}~and~\ref{def:agnostic}.
Indeed, those two definitions are logically equivalent to each other.
One could almost say that Steve's demonstration that those two are logically equivalent suggests that there is something wrong with at least one of the definitions.
Steve's argument is logically sound, but it doesn't tell us about agnosticism in a sense that anyone really cares about.

In definition~\ref{def:love} I give a ridiculous definition of \rllove.
It is, however, a definition, and I have some freedom to define terms as I wish as long as I am clear about them.

\begin{definition}[Love {[ridiculous definition]}]\label{def:love}
    \rllove\ is an infinite sum!
    \[
     \rllove = \sum_{n=0}^\infty \frac{1}{n!}   
    \]   
\end{definition}

We can do something similar with \rlhate, as seen in definition~\ref{def:hate}.
Again, I have explicitly given the definition, and I am free to define things in any logically consistent way as long as I am clear and consistent about it.

\begin{definition}[Hate {[ridiculous definition]}]\label{def:hate}
    \rlhate\ is the base of compound interest.
    \[
     \rlhate =  \lim_{n \to \infty}\left( 1 + \frac{1}{n}\right)^n 
    \]   
\end{definition}

From those definitions I could also prove\footnote{%
    Well, I would have to look up the proof.
    I am not capable of reconstructing the proof and I doubt that I ever fully understood it.
    These are different representations of the mathematical constant,
    $e$, the base of the natural logarithm.}
that \rllove\ and \rlhate\ are the very same thing.
My proof would be sound and valid, but it would tell us nothing interesting about love and hate.

If I were to present that as an argument that love and hate were logically indistinguishable I should expect to be laughed at.
Steve's argument, while flawed in the identical way, is not as ridiculous.
My definitions of \rllove\ and \rlhate\ are transparently and deliberately missing the mark on capturing what we might want to capture in a definition of love and hate.
Steve's definition~(\ref{def:agnostic}) of \rlagno\ misses its mark,
but it does so in a far less obvious way.

\section{Right (for the wrong reason)}

Despite my objections in section~\ref{sec:object}, I do agree with Steve that it is silly to claim to be a weak atheist. That is what I will get to in this section when I write it.


\end{document}
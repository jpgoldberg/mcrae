\documentclass[12pt]{article}

\usepackage{amsmath}
\usepackage{booktabs}
\usepackage{csquotes}
\usepackage[standard]{ntheorem}

\providecommand{\land}{\wedge}
\providecommand{\lor}{\vee}
\providecommand{\lif}{\rightarrow}
\providecommand{\liff}{\iff}
\providecommand{\lnot}{\neg}
\providecommand{\lfalse}{\bot}
\providecommand{\lall}{\forall}
\providecommand{\lis}{\exists}
\newcommand{\rel}[1]{\ensuremath{\mathop{\mathsf{#1}}}}

\newcommand{\rlan}{\rel{Ancestor}}
\newcommand{\rlpt}{\rel{Parent}}
\newcommand{\rlapple}{\rel{Apple}}
\newcommand{\rlfruit}{\rel{Fruit}}
\newcommand{\rlbelieve}{\rel{Believe}}

\newcommand{\rlatheist}{\rel{Atheist}}
\newcommand{\rlwa}{\rel{WeakAtheist}}
\newcommand{\rlagno}{\rel{Agnostic}}



\begin{document}

Steve McRea has been attempting to present what should be a simple and uncontroversial logical argument, but he has run into substantial resistance.
I believe that there are two reasons for the resistance, but before discussing those, I want to first make sure that I (and perhaps others) properly understand his argument as he intends it.

\section{Logical notation}

The argument that he has given is presented using propositional symbolic logic.
In recasting it, I will sometimes resort to first order predicate logic, mostly to make it easier to talk about certain parts of the argument, even if it isn't strictly necessary. The notation I will be using for propositional logic is shown in table~\ref{tab:prop}. 

\begin{table}
    \begin{center}
    \begin{tabular}{cc}
        \toprule
        Symbol & Word \\
        \midrule
        $\land$ & and \\
        $\lor$  &  or \\
        $\lnot$ & not \\
        $\lif$  & implies \\
        $\liff$ & if and only if \\
        $p, q, r, \dots$ & various propositions \\
        \bottomrule  
    \end{tabular}
    \caption[Propositional logic symbols]{Symbols for propositional logic with hints at what they kinda-sorta mean in English}\label{tab:prop}
    \end{center}
\end{table}

I am not going to teach symbolic logic, but I will give a couple of examples to help people read those. If we let $p$ stand for the proposition “Jones owns a Ford” and we let the $q$ stand for the proposition that “Brown is in Barcelona”, we could represent “Jones owns a Ford or Brown in in Barcelona” with $p \lor q$.
We can represent more complicated things by building up expressions from the symbols.

Propositional logic only allows for talking about the relationships between simple proposition without being able to look inside them.
But suppose we want to talk about the relationship between notions like $\rlapple$ and $\rlfruit$. So say that some thing, $t$ is an apple we would say $\rlapple(t)$.

We might want to be able to say that if something is an apple it is a fruit.
For this we need a way to talk about all things. And the notation for that is the universal quantifier ($\lall$), which you can read as “for all.” So one way to represent “all apples are fruits” is to say that if for anything that it is an apple that this is also a fruit. And that is what we have in expression \ref{exp:allapple}.

\begin{equation}\label{exp:allapple}
    \lall x \rlapple(x) \lif \rlfruit(x)
\end{equation}

$\rlan$ and $\rlpt$. So if we want to say that Alice, $a$, is a direct ancestor of Bob, $b$, we could notate it with something like “$\rlan(a, b)$”.

We might also want to say things like an apple exists, then a fruit exists.
We can do that using what is called the existential quantifier ($\lis$), which you can read is “there is.” This is exemplified in \ref{exp:isapple}.

\begin{equation}\label{exp:isapple}
    \lis x \rlapple(x) \lif \lis y \rlfruit(y)
\end{equation}

Note that \ref{exp:isapple} doesn't say that the thing that is the apple is the fruit.\footnote{Don't worry about this. It just means I should have found a better example that doesn't add in that extra confusion.}

\section{Steve's first argument}

Steve takes issues with Atheists who like to state that Atheism is a lack of belief.\footnote{I also think that the claim that “Atheism is a lack of belief” is silly, but for very different reasons than Steve does. I am, however, trying to present Steve's argument in this section.}
The guts of the argument is that "lack of a belief about the existence of a god or gods is logically equivalent to agnosticism.

Let's follow Steve's notation and use $g$ to denote the proposition that god (or gods) exist.
I am going to depart from his notation by treating \rlbelieve\ as a two-place relation. So $\rlbelieve(x, p)$ is a way to say that individual $x$ believes proposition, $p$. This departure doesn't change anything of substance (yet), but is perhaps clearer as it corresponds more closely to English “believe”.

What I most notably do differently is make explicit and clear how notions like (strong) Atheist, weak Atheist, and Agnostic are represented. Again, I am attempting to follow Steve's intent.

In Definition~\ref{def:satheist} we define  \rlatheist\ in terms of \rlbelieve\ and $g$. Note that this definition is intended to correspond to what some call a strong Atheist.

\begin{definition}[Atheist]\label{def:satheist}
    An individual, $x$, is an atheist if and only if they believe that the proposition that $g$ that god exist is false.

    \[
        \lall x \rlatheist(x) \liff \rlbelieve(x, \lnot g)
    \]
\end{definition}

The weak Atheist, by comparison neither believe that god(s) exist nor believe that god(s) don't exist. And that is what we intend to capture in Definition~\ref{def:weak}.

\begin{definition}[Weak Atheist]\label{def:weak}
    An individual, $x$, is a weak atheist if and only if they neither believe the proposition, $g$, that god(s) exist nor the proposition, $\lnot g$, that god(s) don't exist.
    \[
        \lall x \rlwa(x) \liff
          \lnot\left[\rlbelieve(x, g) \lor \rlbelieve(x, \lnot g)\right]
    \]
\end{definition}

Although we won't be using this, I will toss in the definition (\ref{def:theist}) for a theist, to server as a simple example of (and provide practice) for reading these sorts of definitions.

\begin{definition}[Theist]\label{def:theist}
    An individual, $x$, is theist if and only if they believe the proposition, $g$, that god(s) exist.
    \[
        \lall x \rel{Theist}(x) \liff \rlbelieve(x, g)
    \]
\end{definition}

Finally, we come to the definition of agnostic within this system. They don't believe that god(s) exist and they don't believe that god(s) don't exist.
This we attempt to capture in Definition~\ref{def:agnostic}.

\begin{definition}[Agnostic]\label{def:agnostic}
    An individual, $x$, is a agnostic if and only if they don't believe proposition, $g$, that god(s) exist and they don't believe the proposition, $\lnot g$, that god(s) don't exist.
    \[
        \lall x \rlagno(x) \liff
          \lnot \rlbelieve(x, g) \land \lnot\rlbelieve(x, \lnot g)
    \]
\end{definition}

Steve's argument (when recast this way) is that \rlwa\ and \rlagno\ are logically equivalent.
And in that he is perfectly correct. We can prove that they are logically equivalent using only the tools of propositional logic.
At a high level there are two steps to the proof. First we use (one half of) De~Morgan's Rule to show that $\lnot \rlbelieve(x, g) \land \lnot\rlbelieve(x, \lnot g)$ is the logically equivalent to $\lnot\left[\rlbelieve(x, g) \lor \rlbelieve(x, \lnot g)\right]$.

\begin{theorem}[De~Morgan's Rule 1]
    For any propositions $p$ and $q$ 
    \[
        \lnot\left(p \lor q\right) \liff \lnot p \land \lnot q
    \]  
\end{theorem}


The second step is to show that if \rlwa\ and \rlagno\ are equivalent to the same thing, then they are equivalent to each other. That is, if $p$ is equivalent to $r$ and $q$ is equivalent to $r$ than $p$ is equivalent to $r$.\footnote{I could provide the full proofs of De Morgan's Rule and of the transitivity of equivalence, but doing so does not help illuminate the argument. Also, I would have to introduce the proof notation.}

Anyway, from the above we can conclude \rlwa\ and \rlagno\ are the very same thing. The argument is sound and valid in logical terms.

\section{The objections}

\relax[To be continued after I confirm with Steve that I have correctly and fairly recast his argument]


\end{document}